%------------------------------------------------%
% 3
Effect size (ES) is a name given to a family of indices that measure the magnitude of
a treatment effect. Unlike significance tests, these indices are independent of sample
size. ES measures are the common currency of meta-analysis studies that summarize
the findings from a specific area of research. See, for example, the influential metaanalysis
of psychological, educational, and behavioral treatments by Lipsey and
Wilson (1993).
There is a wide array of formulas used to measure ES. For the occasional reader of
meta-analysis studies, like myself, this diversity can be confusing. One of my
objectives in putting together this set of lecture notes was to organize and summarize
the various measures of ES.
In general, ES can be measured in two ways:
a) as the standardized difference between two means, or
b) as the correlation between the independent variable classification and the
individual scores on the dependent variable. This correlation is called the "effect size
correlation" (Rosnow & Rosenthal, 1996).

%----------------%
Meta Analysis

A meta-analysis is a summary of previous research that uses quantitative methods to
compare outcomes across a wide range of studies. Traditional statistics such as t tests
or F tests are inappropriate for such comparisons because the values of those
statistics are partially a function of the sample size. Studies with equivalent
differences between treatment and control conditions can have widely varying t and
F statistics if the studies have different sample sizes. Meta analyses use some
estimate of effect size because effect size estimates are not influenced by sample
sizes. Of the effect size estimates that were discussed earlier in this page, the most
common estimate found in current meta analyses is Cohen's d.
