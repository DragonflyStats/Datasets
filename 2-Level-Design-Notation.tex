%% - http://www.itl.nist.gov/div898/handbook/pri/section3/pri32.htm

Notation for 2-Level Designs
Matrix notation for describing an experiment	The standard layout for a 2-level design uses +1 and -1 notation to denote the "high level" and the "low level" respectively, for each factor. For example, the matrix below
\begin{verbatim}
Factor 1 (X1)	Factor 2 (X2)
Trial 1	-1	-1
Trial 2	+1	-1
Trial 3	-1	+1
Trial 4	+1	+1
\end{verbatim}
describes an experiment in which 4 trials (or runs) were conducted with each factor set to high or low during a run according to whether the matrix had a +1 or -1 set for the factor during that trial. If the experiment had more than 2 factors, there would be an additional column in the matrix for each additional factor.

Note: Some authors shorten the matrix notation for a two-level design by just recording the plus and minus signs, leaving out the "1's".

Coding the data

The use of +1 and -1 for the factor settings is called coding the data. This aids in the interpretation of the coefficients fit to any experimental model. After factor settings are coded, center points have the value "0". Coding is described in more detail in the DOE glossary.
