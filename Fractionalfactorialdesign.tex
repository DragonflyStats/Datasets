

%--------------------------------------------------------------------------------------------------%
\section{Fractional factorial design}

(d)	Define the following terms used in fractional factorial design; Defining relation,
Generator, Confounding, Resolution. Which design resolution is considered
optimal?
\section{Factorial Design}
Factorial experiments permit researchers to study behavior under conditions in which independent variables, called in this context factors, are varied simultaneously.

Thus, researchers can investigate the joint effect of two or more factors on a dependent variable. The factorial design also facilitates the study of interactions, illuminating the effects of different conditions of the experiment on the identifiable subgroups of subjects participating in the experiment.


A full factorial experiment is an experiment whose design consists of two or more factors, each with discrete possible values or ``levels", and whose experimental units take on all possible combinations of these levels across all such factors. A full factorial design may also be called a fully-crossed design. Such an experiment allows studying the effect of each factor on the response variable, as well as the effects of interactions between factors on the response variable.

For the vast majority of factorial experiments, each factor has only two levels. For example, with two factors each taking two levels, a factorial experiment would have four treatment combinations in total, and is usually called a $2\times2$ factorial design.

\newpage

