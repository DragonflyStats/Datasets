\documentclass[12pt]{article}

\usepackage{framed}
\usepackage{amsmath}
\usepackage{amssymb}
\usepackage{graphics}

\voffset=-1.5cm
\oddsidemargin=0.0cm
\textwidth = 470pt
\begin{document}
\section*{Factorial Design For Experiments}


\subsection*{Factorial Designs}

A factorial design is the most common way to study the effect of two or more independent variables.   In a factorial design, all levels of each independent variable are combined with all levels of the other independent variables to produce all possible conditions.  

%For example, a researcher might be interested in the effect of whether or not a stimulus person (shown in a photograph) is smiling or not on ratings of the friendliness of that person.  The researcher might also be interested in whether or not the stimulus person is looking directly at the camera makes a difference.  In a factorial design, the two levels of the first independent variable (smiling and not smiling) would be combined with the two levels of the second (looking directly or not) to produce four distinct conditions: smiling and looking at the camera, smiling and not looking at the camera, not smiling and looking at the camera, and not smiling and not looking at the camera.

A \textbf{2x2} (two-by-two) factorial design is an experiment are two independent variables, each of which has two levels.  If one of the independent variable had three levels , then it would be a \textbf{3x2} factorial design.  Note that the number of distinct conditions formed by combining the levels of the independent variables is always just the product of the numbers of levels.  In a 2x2 design, there are four distinct conditions.  In a 3x2 design, there are 6.

\subsection*{Example of 2x2 Design}
For instance, testing aspirin versus placebo and clonidine versus placebo in a randomized trial. Each patient is randomized to (clonidine or placebo) and (aspirin or placebo). The main effect of aspirin and the main effect of clonidine on the outcome of interest can be assessed using a two-way ANOVA.

This trial design is useful to detect an interaction (this is where the effect on the outcome of one factor (e.g. aspirin) depends on the level of the other factor (i.e. whether or not the person gets clonidine), but one must be careful, as many factorial trials are not powered to detect an \textit{\textbf{interaction}}. Therefore, one runs the risk of falsely declaring that there is no interaction, when in fact there is one (a type II error).

%Therefore, I wouldn't say the two have the same assumptions, as one is a design and one is a statistical method. That being said, the two-way ANOVA is a great way of analyzing a 2x2 factorial design, since you will get results on the main effects as well as any interaction between the effects.

\subsection*{Interaction Effects}
 
An interaction effect is a special kind of effect that can be observed in factorial experiments.  You have an interaction whenever the effect of one independent variable depends on the level of the other.  

There is no interaction when the effect of one variable is essentially the same regardless of the level of the other. 

\subsection*{Example of Interaction Effects}
A cholesterol reduction clinic has two diets and one exercise regime. It was found that exercise alone was effective, and diet alone was effective in reducing cholesterol levels (main effect of exercise and main effect of diet). Also, for those patients who didn't exercise, the two diets worked equally well (main effect of diet); those who followed diet A and exercised got the benefits of both (main effect of diet A and main effect of exercise). However, it was found that those patients who followed diet B and exercised got the benefits of both plus a bonus, an interaction effect (main effect of diet B, main effect of exercise plus an interaction effect).

\subsection*{Relationship Between Main Effects and Interactions}
 
In a $2\times 2$ factorial experiment, there are two main effects (one for each independent variable) and one interaction (the one between the two independent variables).  It gets more complicated with more independent variables. 

 An experiment with three independent variables would have three main effects (again, one for each independent variable) and four interactions.  There is an interaction between IV’s 1 and 2, 2 and 3, and 1 and 3.  

There is also a \textit{\textbf{three-way interaction}}, which has to do with whether the interaction between two variables depends on the level of the third.



\end{document}