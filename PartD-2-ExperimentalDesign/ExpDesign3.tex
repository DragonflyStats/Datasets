\documentclass[12pt]{article}
\voffset=-1.5cm
\oddsidemargin=0.0cm
\textwidth = 470pt
\usepackage{framed}
\usepackage{amsmath}
\usepackage{amssymb}
\usepackage{graphics}

%opening
\title{Inference Procedures}
%\author{MA4605}

\begin{document}
\Large
\section*{Experimental Design}

\subsection*{Control}
\begin{itemize}
\item Suppose a farmer wishes to evaluate a new fertilizer. She uses the new fertilizer on one field of crops (A), while using her current fertilizer on another field of crops (B). The irrigation system on field A has recently been repaired and provides adequate water to all of the crops, while the system on field B will not be repaired until next season. She concludes that the new fertilizer is far superior.
\item The problem with this experiment is that the farmer has neglected to \textit{\textbf{control}} for the effect of the differences in irrigation. This leads to \textbf{\textit{experimental bias}}, the favoring of certain outcomes over others. To avoid this bias, the farmer should have tested the new fertilizer in identical conditions to the control group, which did not receive the treatment. 
\item Without controlling for outside variables, the farmer cannot conclude that it was the effect of the fertilizer, and not the irrigation system, that produced a better yield of crops.

\item Another type of bias that is most apparent in medical experiments is the \textit{\textbf{placebo effect}}. Since many patients are confident that a treatment will positively affect them, they react to a control treatment which actually has no physical affect at all, such as a sugar pill. For this reason, it is important to include control, or placebo, groups in medical experiments to evaluate the difference between the placebo effect and the actual effect of the treatment.

\item The simple existence of placebo groups is sometimes not sufficient for avoiding bias in experiments. If members of the placebo group have any knowledge (or suspicion) that they are not being given an actual treatment, then the effect of the treatment cannot be accurately assessed. 
\item For this reason, \textit{\textbf{double-blind experiments}} are generally preferable. In this case, neither the experimenters nor the subjects are aware of the subjects' group status. This eliminates the possibility that the experimenters will treat the placebo group differently from the treatment group, further reducing experimental bias.
\end{itemize}

\newpage
\subsection*{Confounding}
\begin{itemize}
\item \textbf{(IMPORTANT)} Confounding occurs when the experimental controls do not allow the experimenter to reasonably eliminate plausible alternative explanations for an observed relationship between independent and dependent variables.
\end{itemize}

\subsubsection*{Example}
\begin{itemize}
\item Consider this example. A drug manufacturer tests a new cold medicine with 200 participants - 100 men and 100 women. The men receive the drug, and the women do not. At the end of the test period, the men report fewer colds.

\item This experiment implements no controls. As a result, many variables are confounded, and it is impossible to say whether the drug was effective. For example, gender is confounded with drug use. Perhaps, men are less vulnerable to the particular cold virus circulating during the experiment, and the new medicine had no effect at all. Or perhaps the men experienced a placebo effect.

\item This experiment could be strengthened with a few controls. Women and men could be randomly assigned to treatments. One treatment group could receive a placebo, with blinding. Then, if the treatment group (i.e., the group getting the medicine) had sufficiently fewer colds than the control group, it would be reasonable to conclude that the medicine was effective in preventing colds.
\end{itemize}


\subsection*{Randomization}

Because it is generally extremely difficult for experimenters to eliminate bias using only their expert judgment, the use of randomization in experiments is common practice. In a randomized experimental design, objects or individuals are randomly assigned (by chance) to an experimental group. Using randomization is the most reliable method of creating homogeneous treatment groups, without involving any potential biases or judgments. There are several variations of randomized experimental designs, two of which are briefly discussed below.

\subsection*{Completely Randomized Design}

In a completely randomized design, objects or subjects are assigned to groups completely at random. One standard method for assigning subjects to treatment groups is to label each subject, then use a table of random numbers to select from the labelled subjects.

\subsection*{Randomized Block Design}

If an experimenter is aware of specific differences among groups of subjects or objects within an experimental group, he or she may prefer a \textit{\textbf{randomized block design}} to a completely randomized design. In a block design, experimental subjects are first divided into homogeneous blocks before they are randomly assigned to a treatment group. 

If, for instance, an experimenter had reason to believe that age might be a significant factor in the effect of a given medication, he might choose to first divide the experimental subjects into age groups, such as under 30 years old, 30-60 years old, and over 60 years old. Then, within each age level, individuals would be assigned to treatment groups using a completely randomized design. In a block design, both control and randomization are considered.

\subsubsection*{Example}
\begin{itemize}
\item A researcher is carrying out a study of the effectiveness of four different skin creams for the treatment of a certain skin disease. He has eighty subjects and plans to divide them into 4 treatment groups of twenty subjects each. 
\item Using a \textit{\textbf{randomized block design}}, the subjects are assessed and put in blocks of four according to how severe their skin condition is; the four most severe cases are the first block, the next four most severe cases are the second block, and so on to the twentieth block. 
\item The four members of each block are then randomly assigned, one to each of the four treatment groups. 
\end{itemize}

\newpage
\subsection*{Matched Pairs Design}

A matched pairs design is a special case of the randomized block design. It is used when the experiment has only two treatment conditions; and participants can be grouped into pairs, based on some blocking variable. Then, within each pair, participants are randomly assigned to different treatments.

The table shows an example of matched pairs design experiment. The 1000 participants are grouped into 500 matched pairs. Each pair is matched on gender and age. For example, Pair 1 might be two women, both age 21. Pair 2 might be two women, both age 22, and so on.

\begin{center}
\begin{tabular}{|c||c|c|}
\hline   
Pair	& Placebo &	Vaccine \\ \hline \hline
1	&1	&1 \\ \hline
2	&1	&1 \\ \hline
$\ldots$	& $\ldots$	& $\ldots$ \\ \hline
499	&1	&1 \\ \hline
500	&1	&1 \\ \hline
\end{tabular} 
\end{center}
For this example, the matched pairs design is an improvement over the completely randomized design and the randomized block design. Like the other designs, the matched pairs design uses randomization to control for confounding. However, unlike the others, this design explicitly controls for two potential lurking variables - age and gender.

\end{document}