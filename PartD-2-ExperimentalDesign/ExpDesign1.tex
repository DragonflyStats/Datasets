\documentclass[12pt]{article}

\usepackage{framed}
\usepackage{amsmath}
\usepackage{amssymb}
\usepackage{graphics}

%opening
\title{Experimental Design}
%\author{MA4605}

\begin{document}
\section*{Experimental Design}
\begin{itemize}
\item An \textit{\textbf{experiment}} is a process or study that results in the collection of data. The results of
experiments are not known in advance. Usually, statistical experiments are conducted in
situations in which researchers can manipulate the conditions of the experiment and can
control the factors that are irrelevant to the research objectives. 
%For example, a rental car company compares the tread wear of four brands of tires, while also controlling for the type ofcar, speed, road surface, weather, and driver.

\item \textit{\textbf{Experimental design}} is the process of planning a study to meet specified objectives. Planning an experiment properly is very important in order to ensure that the right type of data and a sufficient sample size and power are available to answer the research questions of interest as
clearly and efficiently as possible.

\item \textit{\textbf{Blocks}} are groups of experimental units that are formed to be as homogeneous as possible with respect to the block characteristics. The term block comes from the agricultural heritage of experimental design where a large block of land was selected for the various treatments, that had uniform soil, drainage, sunlight, and other important physical characteristics. Homogeneous clusters improve the comparison of treatments by randomly allocating levels of the treatments within each block.
\end{itemize}
\newpage
\subsection*{Types of Effects}
\begin{description}
\item[Effect]
An effect is a change in the response due to a change in a factor level. There are different
types of effects. One objective of an experiment is to determine if there are significant
differences in the responses across levels of a treatment (a fixed effect) or any interaction
between the treatment levels. If this is always the case, the analysis is usually easily
manageable, given that the anomalies in the data are minimal (outliers, missing data,
homogeneous variances, unbalanced sample sizes, and so on).
\item[Random Effect]
A random effect exists when the levels that are chosen represent a random selection from a
much larger population of equally usable levels. This is often thought of as a sample of
interchangeable individuals or conditions. The chosen levels represent arbitrary realizations
from a much larger set of other equally acceptable levels.
\end{description}

\end{document}