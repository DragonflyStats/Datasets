
%--------------------------------------------------------%
\section{Orthogonal Array}
Orthogonal array testing is a systematic, statistical way of testing. Orthogonal arrays can be applied in user interface testing, system testing, regression testing, configuration testing and performance testing.
All orthogonal vectors exhibit orthogonality. Orthogonal vectors exhibit the following properties:
Each of the vectors conveys information different from that of any other vector in the sequence, i.e., each vector conveys unique information therefore avoiding redundancy.
On a linear addition, the signals may be separated easily.
Each of the vectors is statistically independent of the others.
When linearly added, the resultant is the arithmetic sum of the individual components.


\subsection{Two Factor Interaction}
The effects of interest in the $2^2$ design are the  main effects A and B and the two-factor interaction AB. It is easy to estimate the effects of these factors.

\begin{eqnarray}
A = \frac{a+ab}{2n} -  \frac{b + (1)}{2n}\\
B = \frac{b+ab}{2n} -  \frac{a + (1)}{2n}\\
AB = \frac{ab + (1)}{2n} -  \frac{a + b}{2n}\\
\end{eqnarray}

The Sums of Squares formulae are
\begin{eqnarray}
SS_{A} = \frac{[(a + ab)-(b + (1))]^2}{4n^2}\\
SS_{B} = \frac{[(b + ab)-(a + (1))]^2}{4n^2}\\
SS_{AB} = \frac{[(ab + (1))-(a + b)]^2}{4n^2}\\
\end{eqnarray}

%---------------------------------------------------------------------%
%---------------------------------------------------------------------%
%---------------------------------------------------------------------%

